\documentclass[screen,anonymous,review,natbib=false]{acmart} %% draft
%%
%% \BibTeX command to typeset BibTeX logo in the docs
\AtBeginDocument{%
  \providecommand\BibTeX{{%
    \normalfont{} B\kern-0.5em{\scshape i\kern-0.25em b}\kern-0.8em\TeX}}}

\acmDOI{...} % DOI - Insert your DOI below...
\acmISBN{...} % ISBN - Insert your conference/workshop's ISBN below...
\acmPrice{15.00}
\acmConference[Short Name]{Long Name}{dates}{venue}
\setcopyright{acmcopyright}
\copyrightyear{2020}
\acmYear{2020}
\acmDOI{}

\PassOptionsToPackage{final}{microtype}
\PassOptionsToPackage{usenames,dvipsnames,svgnames,table,x11names}{xcolor}
\PassOptionsToPackage{cache=false}{minted}
\usepackage{fix-cm}
\usepackage{graphicx, color} 
\usepackage{fontspec,newunicodechar}
\usepackage{graphbox} %% an extension of graphix, for \includegraphics 
\usepackage{hyphenat} %% electromagnetic\hyp{}endioscopy
\usepackage[shortcuts]{extdash}
\usepackage{polyglossia} %% for xelatex use polyglossia
\setdefaultlanguage[variant=american]{english}
%% \usepackage{alltt} %% Commented in favor of fancyvrb, which minted loads
\usepackage{chngcntr} %% for changing counters in fixing the way sections, etc print.
\usepackage{hang} %% to include hanging indent packages, AFAIK better than {hanging}
\defaultfontfeatures{Ligatures=TeX,Mapping=tex-text}
% Needed for the bibliography commands
% ^Needed to build bib entries in text, outside bibliography
%% \usepackage{pdftexcmds} %% useful for LuaTex, not needed here
%% \usepackage{bibentry} %% For inserting full bib citations into the document. not needed w/biblatex
\usepackage{mathtools} %% loads amsmath w/improvements, which needs to be before hyperref
%% \usepackage{mathspec} -- causes a clash; subsumes fonstpec, must come after amsmath
\usepackage{latexsym} %% additional symbols
%%\usepackage[warnings-off={mathtools-colon}]{unicode-math} -- can't be added, b/c it overwrites \eth
% \usepackage{amssymb}  %% -- can't be added, b/c it overwrites \Bbbk 
\usepackage{wasysym}  %% additional symbols
\usepackage{bold-extra} %% \textsc{ \textbf{This is bold small capitals} }
\usepackage{tabu} %% the currently-preferred table package
%% But! See: https://tex.stackexchange.com/questions/121841/is-the-tabu-package-obsolete
\usepackage{pdflscape} %% for \begin{landscape} ..., to change part of doc to landscape
%% \usepackage{multicol} %% provides multicol environment
\usepackage{morefloats} %% raise limit for max # floats above 18
\usepackage{placeins} %% gives \FloatBarrier, which limits float movements [section]
\usepackage{needspace} %% prevents page breaks right after section names
\usepackage{wrapfig} %% adds \begin{wrapfigure}[lh]{pos}[ohang]{w}
% \usepackage{floatrow} %% improved version of sidecap, for horizontal caption placement
\usepackage{etaremune} %% rev enumerate. obv :D
\usepackage{mathrsfs}
\usepackage{calrsfs}
\usepackage{layout}
%\usepackage{showframe} Turn on to see the frame around the stuff 
\usepackage{mathalfa} %% https://tex.stackexchange.com/questions/58098/what-are-all-the-font-styles-i-can-use-in-math-mode/58124#58124
\usepackage{bbold}
%% eufrak package is redundant b/c amsmath
\usepackage{bigintcalc}
\usepackage{ifdraft}
\usepackage{zref}
\usepackage{classlist}
\usepackage{dsfont}
\usepackage{subcaption} %% allows subcaptions before captions 
\usepackage[bottom]{footmisc} %% for customizing footnotes
%\usepackage{varioref} %% adds \vref, w/context sensitivity referencing
\usepackage[all]{hypcap} %% resolves issues w/captions & hyperref. Load after hyperref
\usepackage[capitalize,nameinlink]{cleveref} %% adds \crefrange{}{} and \cpagerefrange{}
\RequirePackage[backend=biber,%
                abbreviate=true,%
                dateabbrev=true,%
                isbn=true,%
                doi=true,%
                urldate=comp,%
                url=true,%
                maxcitenames=2,%
                maxbibnames=4,%
                backref=false,%
                style=ACM-Reference-Format,%
                natbib=true,
                autocite=plain,%
                block=space,%
                alldates=terse,%
                isbn=false,%
                defernumbers=true,%
                sortcites=true,%
                sortsets=true,%
                eprint=false,%
                language=american]{biblatex}
\usepackage{nag} %% should complain about old and outdated commands
\usepackage{mdframed} %% adds \begin{framed} environment, and other simple framing boxes
\usepackage{fancybox} %% adds \doublebox, \ovalbox, and \shadowbox 
%% must load fancybox before minted
\usepackage{xpatch} %% for etoolbox w/biblatex
\usepackage{minted} %% My brand new baby for highlighting source code
%% also adds \begin{framed} environment
\usepackage{mathpartir} %% Didier Remy's typesetting for proofs. 
\usepackage[strict,autostyle]{csquotes} %% enquote &c that doesn't interfere w/cdlatex usage http://ftp.lyx.org/pub/tex-archive/macros/latex/contrib/csquotes/csquotes.pdf
\usepackage[xspace]{ellipsis} %% http://tex.stackexchange.com/questions/73467/what-is-best-method-to-insert-an-ellipsis-in-a-xetex-document#73469
\usepackage{pseudocode}
\usepackage{epigraph}
\usepackage{algorithmicx}
\usepackage[noend]{algpseudocode}
\usepackage{tikz-cd}
\usepackage{algorithm} %% must come after floatrow [chapter] 
\usepackage[defaultlines=3,all]{nowidow} %% Eliminates widow and orphans 
\addbibresource{streams.bib}

\setmonofont{DejaVuSansMono}[Scale=MatchLowercase] %% What is this scale option?
\tracinglostchars=2

\newmintinline[rackinline]{racket}{}
\newmintinline{prolog}{}

%% defines posscite, \posscites, \Posscites
%% From https://tex.stackexchange.com/a/307461/109698
\DeclareNameFormat{labelname:poss}{% Based on labelname from biblatex.def
  \nameparts{#1}% Not needed if using Biblatex 3.4
  \ifcase\value{uniquename}%
    \usebibmacro{name:family}{\namepartfamily}{\namepartgiven}{\namepartprefix}{\namepartsuffix}%
  \or%
    \ifuseprefix%
      {\usebibmacro{name:first-last}{\namepartfamily}{\namepartgiveni}{\namepartprefix}{\namepartsuffixi}}
      {\usebibmacro{name:first-last}{\namepartfamily}{\namepartgiveni}{\namepartprefixi}{\namepartsuffixi}}%
  \or%
    \usebibmacro{name:first-last}{\namepartfamily}{\namepartgiven}{\namepartprefix}{\namepartsuffix}%
  \fi
  \usebibmacro{name:andothers}%
  \ifnumequal{\value{listcount}}{\value{liststop}}{'s}{}}
\DeclareFieldFormat{shorthand:poss}{%
  \ifnameundef{labelname}{#1's}{#1}}
\DeclareFieldFormat{citetitle:poss}{\mkbibemph{#1}'s}
\DeclareFieldFormat{label:poss}{#1's}
\newrobustcmd*{\posscitealias}{%
  \AtNextCite{%
    \DeclareNameAlias{labelname}{labelname:poss}%
    \DeclareFieldAlias{shorthand}{shorthand:poss}%
    \DeclareFieldAlias{citetitle}{citetitle:poss}%
    \DeclareFieldAlias{label}{label:poss}}}
\newrobustcmd*{\citeposs}{%
  \posscitealias%
  \textcite}
\newrobustcmd*{\Citeposs}{\bibsentence\citeposs}
\newrobustcmd*{\Citeposses}{%
  \posscitealias%
  \textcites}

\newcommand{\meta}[2]{[\textsc{#1}: \emph{#2}]}
\newcommand{\todo}[1]{\textcolor{red}{\meta{TODO}{#1}}}

%%
%% end of the preamble, start of the body of the document source.
\begin{document}
\title{The Cost of Occurs-check in Relational Program Synthesis}
%\title{Improved Performance of Relational Synthesis through Eliminating Occurs Checks}
%\title{The Impact of Unification Algorithms in Relational Synthesis Applications} 
%\title{An Optimizing miniKanren for more Performant Relational Synthesis}

\author{Us}
\email{void@example.com}
\orcid{1234-5678-9012}
\affiliation{%
  \institution{Northeastern University}}


%% By default, the full list of authors will be used in the page
%% headers. Often, this list is too long, and will overlap
%% other information printed in the page headers. This command allows
%% the author to define a more concise list
%% of authors' names for this purpose.
% \renewcommand{\shortauthors}{}

\begin{abstract}
\end{abstract}

\begin{CCSXML}
<ccs2012>
   <concept>
       <concept_id>10003752.10003790.10003795</concept_id>
       <concept_desc>Theory of computation~Constraint and logic programming</concept_desc>
       <concept_significance>500</concept_significance>
       </concept>
 </ccs2012>
\end{CCSXML}

\ccsdesc[500]{Theory of computation~Constraint and logic programming}

%%
%% Keywords. The author(s) should pick words that accurately describe
%% the work being presented. Separate the keywords with commas.
\keywords{relational, program synthesis, unification, miniKanren, logic programming}


%%
%% This command processes the author and affiliation and title
%% information and builds the first part of the formatted document.
\maketitle

\section{Introduction}\label{sec:intro}

\section{Background}\label{sec:background}


\todo{What about that linear one designed to be more efficient in
  practice? Add that one in here.}

\section{Theory}\label{sec:theory}

\begin{listing}
\begin{minted}[autogobble,stripall,linenos]{racket}  
(define welcome
  (λ (x)
    (printf "hello ~s~n" x)))
\end{minted}
\caption{The \rackinline|welcome| method says hello to the argument.}
\label{mnt:welcome}
\end{listing}

\section{Performance}\label{sec:performance}

\begin{verbatim}
In generating 4 thrines (fresh), we had the following
succeed-ct: 216376
clash-ct: 97712
cycle-ct: 6892
\end{verbatim}


The numbers for the more complicated queries from
\citet{byrd2017unifiedapproach} are also interesting to see.

It would be interesting to know the ``run lengths'' of these successes
and either kind of failure as we compute, either in between
interleaves or in each interleave. For the \citet{byrd2017unifiedapproach}
final interpreter example. It seems that in practice on many examples
\citeposs{robinson1965machine} version out-performs the
asymptotically-superior algorithms.

\href{file://~/Downloads/icfp-17/shared/}{file://~/Downloads/icfp-17/shared/}
\todo{Run these same performance comparisons against the Nada quines problem from TLS, without using any constraints but ==.}

\section{Related}\label{sec:related}


\section{Conclusions}\label{sec:conclusions}

%%
%% The acknowledgments section is defined using the "acks" environment
%% (and NOT an unnumbered section). This ensures the proper
%% identification of the section in the article metadata, and the
%% consistent spelling of the heading.

% \begin{acks}
%   We thank you all. 
% \end{acks}

\printbibliography{}

\end{document}
\endinput

%%% Local Variables:
%%% mode: latex
%%% TeX-master: t
%%% End:
